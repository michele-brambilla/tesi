
\chapter{Si detectors in space}
\label{sec:si_spazio}

Instrumentation for particle and high-energy photon measurements in space must
provide high levels of performance while meeting the severe constraints imposed
by flight. Direct measurements are required spanning over thirteen decades in
energy and covering species ranging from photons to the heaviest nuclei in the
periodic table. Indirect measurements increase the energy range by another five
decades.  Instruments to measure high-energy photons, X-rays, $\gamma$-rays and
energetic particles are key tools in modern astronomy and
astrophysics. High-energy photons are produced by a wide variety of processes in
which particles, particularly electrons, are accelerated to relativistic
velocities, or in which material is elevated to extreme temperatures. The
particle acceleration processes that produce high-energy photons are also likely
sources of highly energetic particles. Detected at velocities approaching the
speed of light, these particles, known as cosmic rays, include atomic nuclei and
electrons, as well as positrons and antiprotons. The instrumentation required
varies greatly depending on the energy and species to be observed (from X-rays
of $\sim 100$~eV to particles near the upper
limit of the cosmic-ray spectrum at about $10^{15}$~eV).\\
Designers of space instrumentation face a number of special challenges. Whether
for balloons, sounding rockets, or satellites, instruments must conform to
strict weight, dimension, and power limits. Size is a particular issue. Larger,
heavier instruments cost more to build and test. For space-based instruments,
higher weight also demands more powerful launch vehicles with accompanying,
often
dramatic, increases in cost.\\
Power is usually supplied by photovoltaic arrays that have limited area,
supplemented by batteries whose weight has to be considered. Heat generated by
the electronics of flight instruments must be dissipated by radiation to space
because the atmosphere is too thin to support convective cooling. At the same
time, instruments must contend with the heat load of exposure to unobstructed
sunlight. Instruments must be reliable and largely autonomous while
incorporating the versatility to allow reconfiguration on command to change
operational modes or compensate for degradation. Space-based instruments must
contend with the rigors of the launch environment
including shock, vibration, and acoustic loads.\\

These constraints suggest that keep weight and power consumption under control
is a serious concern when design instruments for space physics. A practical
example is given from \gls{egret} and \gls{fermi}, two instruments devoted to
the study of gamma-ray sky. 



\subsection{Si detector in space and track reconstruction}
\label{sec:tracking_spazio}

%Double sided silicon detectors have been also applied in space
%experiments where double sided silicon detector are well suited to the
%purpose of providing two coordinates per detector and an optimal
%spatial resolution keeping multiple scattering at a minimum level.\\

For energies below a few hundred MeV/nucleon, one of the most common
techniques used for the identifcation of cosmic rays is the
simultaneous measurement of specifc ionization energy loss, dE/dx, and
total kinetic energy, E. This technique is capable of determining the
charge, mass and energy of incident particles. In fact the quantity 
\begin{equation}
  \left( \frac{dE}{dx} \right)\left( E \right) \approx \left( \frac{Z^2}{v^2} \right)\left( \frac{1}{2}mv^2 \right)= \frac{1}{2}mZ^2 
\end{equation}
is unique for each completely ionized nuclei and cosmic particles.\\

\subsubsection*{IMP-8}
\hskip 1cm The modern dE/dx-E telescope based on silicon detector was
introduced on the IMP-8 satellite, launched in 1973. IMP-8, a
drum-shaped spacecraft of 135.6~cm diameter and 157.4~cm high, was
instrumented for interplanetary, magnetotail, and magnetospheric
boundary studies of cosmic rays, energetic solar particles, plasma,
and electric and magnetic fields. The spacecraft had a launch mass of
410~kg and a nominal power of 150~W. Among the instruments installed
on the satellite, \gls{epe}, \gls{cpme} and \gls{che} were focused on cosmic rays detection.\\
The \gls{che} (also referred to as \gls{crne}), devoted to the study
of solar flare isotopes was an example of dE/dx vs E telescope. The
instrument consisted of a pair of solid-state telescopes. The main
telescope measures nuclei in the energy range of 10 to 100's of MeV/n,
and electrons in the range of about 2 to about 25 MeV. The second
telescope measures protons and alpha particles in the 0.5-1.8 MeV/n
range.

The general layout of the telescope is shown in \fig{imp_satellite}.
\begin{figure}
  \centering
  \subfloat[\emph{}\label{imp_c}]
  {\includegraphics[width=0.3\textwidth]{cap1/imp_satellite.jpg}}
  \subfloat[\emph{}\label{imp_a}]
  {\includegraphics[width=0.3\textwidth]{cap1/imp_a.jpg}} \\
  \caption{\textit{(a) Photo of the IMP-8 spacecraft and (b) cross
      section of the IMP-8 telescope. The silicon detectors D1, D2, D3
      measure the particle energy
      loss~\cite{Spurio}.}}\label{imp_satellite}
\end{figure}
It consisted of three lithium-drifted silicon detectors (D1, D2, D3)
of thickness 750, 1450 and 800~$\mu$m plus an 11.5~g$\cdot$cm$^{-2}$
thick CsI(Tl) scintillator (D4) viewed by four photodiodes, a sapphire
scintillator/Cerenkov radiator (D5) of thickess 3.98~g$\cdot$cm$^{-2}$
and a plastic scintillation guard counter (D6) viewed by a
photomultiplier
tube.\\
Particles passing through the detector layers D1, D2, D3, and which
have come to rest in D4 are considered. Counters D5 and D6 act as
veto, to confirm that the particle is stopped in D4. The separation
between different chemical species and isotopes is achieved using the
dE/dx technique (measured in D1, D2) as a function of the total energy
released in D4. \fig{imp_b} shows how events due to different isotope
are located along distinct lines that are approximately equilateral
hyperbolae.
\begin{figure}[!htbp]
  \centering
  \includegraphics[width=0.4\textwidth]{cap1/imp_b.jpg}
  \caption{\textit{Result from IMP-8, data collected between 1974 and
      1975. On the y-axis, the signal is proportional to the energy
      loss $ (Z/v)^2$ measured in (D1 + D2), on the x-axis, to the
      residual energy of the particle measured in D4. At a given
      energy, the value on the y-axis increases as $Z^2$. For a given Z
      and energy, nuclei with larger A have smaller $\beta$ and undergo
      larger energy losses in D4~\cite{Spurio}.}}\label{imp_b}
\end{figure}
Nuclei with different Z are well separated by their energy loss, which
depends on $(Z/v)^2$ of the particle. At a given total energy,
proportional to the signal D4, of a nucleus with a given Z, the
velocity $v = \beta c$ is smaller for the isotope with a larger A.\\

\subsubsection*{ISEE-3}

\hskip 1cm Still making use of the dE/dx telescope was the spacecraft
\gls{isee}-3, launched by \gls{nasa} on August 12, 1978. Among the
reasons of the relevance of the experiment is that \gls{isee}-3 is the
first spacecraft injected in a halo orbit around the sunward libration
point L1 of the earth-sun system at about 230 earth radii (\fig{orbit_isee}).
\begin{figure}[!htbp]
  \centering
  \includegraphics[width=0.6\textwidth]{cap1/orbit_isee.jpeg}
  \caption{\textit{ISEE-3 transfer trajectory to the halo orbit~\cite{}.}}\label{orbit_isee}
\end{figure}
%This orbit is ideal for interplanetary
%studies since it is permanently in the solar wind and is sufficiently
%close to the earth to maintain a high bit-rate for data transmission
%and yet far from the interference with earth's magnetosphere.
Goal of the project was the study ofcosmic rays and particles emitted
during solar flares in particular in the interplanetary region near 1
AU \footnote{The astronomical unit AU is a unit of length, roughly the
  distance from Earth to the Sun, about 150 million kilometres.}.  The
experiment consisted of two pairs of telescopes referred to as the
\gls{vlet}'s and the \gls{het}'s. Both the \gls{vlet} and the
\gls{het} systems are constructed out of solid-state detectors because
of the excellent resolution, long
reliability and low weight.\\
\begin{table}[!htbp]
  \centering
  \caption{Area and thickness of the ISEE-3 VLET (a) and HET (a) detectors.}
  \subfloat[%Area and thickness of the VLET detectors.
  \label{tab:isee3-vlet}]
  {
    \begin{tabular}[!htb]{r|cc}
      \hline
      detector  & area       & thickness\\
      \hline
      {\it D}I  & 200 mm$^2$ & 15~$\mu$m  \\
      {\it D}II & 200 mm$^2$ & 15~$\mu$m  \\
      {\it E}   & 425 mm$^2$ & 320~$\mu$m \\ 
      {\it F}   & 425 mm$^2$ & 320~$\mu$m \\ 
      \hline
    \end{tabular}
%    \caption{Area and thickness of the VLET detectors.}\label{tab:isee3-vlet}
  }
  \vspace{.5cm}
  \subfloat[%Area and thickness of the HET detectors.
  \label{tab:isee3-het}]
  {
    \begin{tabular}[!htb]{r|cc}
      \hline
      detector   & area       & thickness\\
      \hline
      A$_{1,2}$   & 800~mm$^2$ & 150~$\mu$m  \\
      B$_{1,2}$   & 800~mm$^2$ & 2000~$\mu$m \\ 
      C$_1$      & 950 ~m$^2$ & 3000~$\mu$m \\ 
      C$_{2,3,4}$ & 950~mm$^2$ & 6000~$\mu$m \\ 
      \hline
    \end{tabular}
  }
\end{table}

The \gls{vlet} consists of four detectors
(\fig{fig:isee3-vlet}): the outer, named {\it D}I, is protected
by a thin foil of Kapton, in order to provide thermal protection,
prevent the detector from detecting solar photons and provide
radiofrequency shielding. The combination of signals from {\it D}I,
{\it D}II and {\it E} allow discrimination of particles, which come to
rest either in {\it D}II or {\it E}. Particles which enter the
detector {\it F} are ignored: it acts as a {\em veto}. Dimensions of
each detector are presented in ~\tab{tab:isee3-vlet}.\\
The \gls{vlet} system weighs 2.63~kg and consumes 2.1~W.\\
\begin{figure}
  \centering
  \subfloat[\emph{}\label{fig:isee3-vlet}]
  {\includegraphics[width=0.4\textwidth]{cap1/isee3-vlet.jpg}}
  \subfloat[\emph{}\label{fig:isee3-het}]
  {\includegraphics[width=0.4\textwidth]{cap1/isee3-het.jpg}} \\
  \caption{\textit{Schematic views of the ISEE3 VLET (a) and HET (b)}}\label{isee3-telescopes}
\end{figure}

The \gls{het} is designed to measure the energy spectra of electrons
and all elements from hydrogen to iron. It is composed of two thin
(150~$\mu$m) surface barrier detectors named A$_1$ and A$_2$,
two curved Li-drifted detectors (B$_1$ and B$_2$) and four
double-grooved Li-drifted detectors (C$_1$ to C$_4$). The double
grooves in the C detectors constitute an anticoincidence
detector denoed G.\\
\fig{fig:isee3-het} shows the structure of the \gls{het}
together with three different kind of events: ``A-Stopping''
(trajectory 1), ``B-Stopping'' (trajectory 2),``Penetrating''
(trajectory 3).\\
The \gls{het} system weighs 4.38~kg and consumes 3.1~W.\\

%The detectors 
%consisted of two Very Low Energy Telescope (VLET's) and two High Energy
%Telescopes (HET's). VLET's had very thin solid-state detectors each with
%an area of 2 cm 2 and a thickness of 15 /lm which allowed measurements of
%all elements from Z = 2 to 26, with energies above 2 MeV IN. The detectors
%aboard ISEE-3 measured solar particle intensities in three energy intervals
%of 2-3, 3.9-6.7 and 6.7-12.4 MeV IN for the major elements, He, 0, Fe and
%others continuously for several days since the flare onset of the September
%23, 1978 and November 11, 1978 flares. New results on the variations of the
%composition ratios of Fe/O and He/O as a function of time were observed.
%Similar measurements were carried out with the ISEE-l spacecraft. Solar
%particle experiments in ISEE-l were conducted by the Max-Planck Institute
%in Garching, Germany in collaboration with five US universities of Arizona,
%New Hampshire, Maryland and Chicago(22) while ISEE-3 experiments were
%done by the NASA-Goddard Space Flight Center(23).\\
Among the experiment achievements
\begin{itemize}
\item \gls{isee}-3 was the first artificial object placed in a halo
  orbit about the Sun-Earth L1 point, proving that such a suspension
  between gravitational fieds was possible;
\item \gls{isee}-3 became the first spacecraft ever to encounter a
  comet. The spacecraft traversed the plasma tail of Comet
  Giacobini-Zinner on September 11, 1985, and made in situ
  measurements of particles, fields, and waves;
\item \gls{isee}-3 transited between the Sun and Comet Halley in late March
1986 and it became the first spacecraft to directly investigate two
comets.
\end{itemize}


\subsubsection*{VOYAGER 1 and 2}

Voyager 1 and 2 spacecraft are mostly known to the broad public
because of the message identifying their time and place of origin for
the benefit of any other spacefarers that might find them in the
distant future. Despite of this the original goal of the mission was
conduct closeup studies of Jupiter and Saturn,
Saturn's rings, and the larger moons of the two planets.\\
Launched in the summer of 1977 and originally built to last 5 years
they are still sending informations during their journey toward
interstellar space. As 17th February 1998 Voyager 1 passed Pioneer 10
to become the most distant
human-made object in space.\\
In 1989 the primary mission - planetary exploration - was completed
and its extension, the \gls{vim} begun. The mission objective of the
\gls{vim} is to obtain useful interplanetary, and possibly
interstellar, fields, particles, and waves (FPW) science data. Among
the currently five science investigation teams participating in the
\gls{vim} two of them make use of silicon-based telescopes
(\fig{vojager}):
\begin{itemize}
\item \gls{lecp} that contained detector configurations designed to
  covere the energy range $15~keV \le E \le 40~MeV/nucleon$;
\item \gls{crs} that contained 46 silicon solid state detectors
  arranged in seven telescopes to determine particle charge and energy
  in the range $1 \le Z \le 26$ and $1~MeV$ to $500~MeV/nucleon$.
\end{itemize}
\begin{figure}
  \centering \subfloat[\emph{}\label{fig:lecp}]
  {\includegraphics[width=0.2\textwidth]{cap1/lecp.jpg}}
  \subfloat[\emph{}\label{fig:vojager_1}]
  {\includegraphics[width=0.6\textwidth]{cap1/vojager_1.jpg}} \\
  \caption{\textit{A picture of the first flight unit of the LECP
      experiment (a) and schematic diagram of the LEPT assembly
      (b).~\cite{}}}\label{vojager}
\end{figure}
%begin{figure}[!htbp]
% \centering
% \includegraphics[width=0.6\textwidth]{cap1/vojager_1.jpg}
% \caption{\textit{Schematic diagram of the LEPT
%     assembly.~\cite{}}}\label{vojager}
%end{figure}
The \gls{lecp} investigators make use of a \gls{lecp} subsystem. It is
an array of solid state detectors designed to measure charge and
energy distribution of low and medium energy nuclei. Particular
emphasis has been placed on acheving the lowest energy respone
attainable with solid state detectors:
\begin{itemize}
\item the first detector D1 consisted of a set of two thick (from 2 to
  5~$\mu$m) surface barrier detectors. The signals from each of the
  units are independently amplified, fed to several discriminators and
  sampled by a 256 channel pulse-height analyzer;
\item the D2 had a large area of $8~cm^2$ and was thick about
  $150~\mu m$ to serve as the total E detector for particles
  penetrating D1. It operated in anticoincidence with detectors A1 to
  A8, and in coincidence with detector D1;
\item detector D5 is $\sim 90~\mu m$ thick and is used as a $\Delta E$
  detector (\fig{lept}) in the medium energy end of the \gls{lept}. In
  combination with D3 and D4, that was used to extend the energy range
  of the telescope, the output provided information for the pulse
  height analyses.
\end{itemize}
\begin{figure}[!htbp]
  \centering
  \includegraphics[width=0.6\textwidth]{cap1/lept.jpg}
  \caption{\textit{The expected $\Delta E$ signals from the D5, D4
      combination.~\cite{}.}}\label{lept}
\end{figure}











% http://voyager.jpl.nasa.gov/index.html
% http://voyager.jpl.nasa.gov/mission/fastfacts.html
% http://voyager.jpl.nasa.gov/mission/interstellar.html
% http://voyager.jpl.nasa.gov/science/principal.html
%   http://sd-www.jhuapl.edu/VOYAGER/
%   http://voyager.ftecs.com/publications/krimigis-ssr14.html
%   http://voyager.gsfc.nasa.gov/heliopause/instruments.html













%As an example, some specific challenges that had to be faced in the
%\gls{glast}\footnote{The Gamma-ray Large Area Telescope is a space
%  mission to explore the gamma-ray universe. The telescope has been
%  designed for high sensitivity, high precision gamma-ray detection in
%  space and it contains more than $80$~m$^{2}$ of single-sided
%  AC-coupled silicon detectors.} experiment were~\cite{Sadrozinski_c}:
%\begin{itemize}
%\item the launch on a Delta II rocket with accelerations of
%  $\sim 10$~g requires mechanical design solutions not needed in
%  accelerator based research. Vibration and acoustic shocks have to be
%  included in the designs. Potential failure modes include breakage
%  of detectors, wire bonds and destruction of the trays;
%\item the down-load bandwidth of the instrument limits the average
%  transmitted data rate to about $30$~Hz, while the expected trigger
%  rate from charged cosmic rays is about $5$~kHz: the data reduction
%  is performed in successful triggers;
%\item the electrical power of a satellite mission is limited. The
%  instrument has a power budget of $650$~W, of which less than half is
%  available for the readout electronics, allowing less than $250$~mW
%  for each of the 1 million channels;
%\item the remoteness and forbidding nature of space poses additional
%  experimental complications. A challenging problem arises from the
%  mismatch of the \gls{cte} in the
%  electronics: silicon detectors, lead converter and the structural
%  materials have different \gls{cte}, which leads to very large forces in
%  large temperature excursions, which might be unavoidable during
%  flight;
%\item the radiation levels in space are a strong function of the orbit
%  and in this case are extremely low by \gls{lhc} standards because
%  the low-earth orbit of GLAST ($\sim 550$~km) is below the radiation
%  belts\footnote{A radiation belt is a layer of energetic charged
%    particles that is held in place around a magnetized planet, such
%    as the Earth, by the planet's magnetic field. The Earth has two
%    such belts and sometimes others may be temporarily created. The
%    discovery of the belts is credited to James Van Allen and as a
%    result the Earth's belts bear his name.}. Though the GLAST orbit
%  traverses the South Atlantic Anomaly (SAA), a region of trapped
%  protons and electrons of fairly low momenta, during which the
%  trigger rate is so high that the experiment is shut down.
%\end{itemize}
%




