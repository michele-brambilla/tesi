
\chapter{Multitrack events with calorimeters}
\label{cap:multitraccia}

\begin{itemize}
\item scopo (breve) e informazioni sui dati
\item descrizione del calorimetro
\item quali passaggi dei precenti abbiamo fatto
\item allineamento e risoluzione
\item tracking eventi singoli
\item multitracking (simulato e reale)
\item risultati
\end{itemize}


Reconstructed tracks have been compared with the outcome of an array of
3$\times$3 SiPM connected to a shashlik electromagnetic calorimeter. Goal of the
activity is to correlate and improve reconstructed hit position with the
calorimeter signal. Before proceeding I will remark some features of
electromagnetic calorimeters that has to be taken into account when analysing
data.\\
When passing trough matter electrons can loose energy because of many different
kind of interaction, nevertheless two main regimes can be identified. At low
energies, below $\sim$10 MeV the energy loss is main due to collision with atoms
and molecules of the material through ionisation and thermal excitation. At
medium energies, $\sim$10 MeV to 1~GeV the main couse of energy loss is
bremsstrahlung. As a consequence electrons of sufficiently high energy
($\geq$1~GeV) incident on a block of material produces secondary photons that can
be in turn converted into an electron-antielectron pair. If these secondary
particles have enough energy the process can continue and give rise to an
electromagnetic cascade. The number or produced particles increase until when
their energy falls below a critical energy 
\begin{equation}\label{eq:critical_energy}
\epsilon \simeq \frac{610 {\rm MeV}}{Z + 1.24}
\end{equation}
thereafter the energy is
released in form of ionisation and excitement.\\
The main features of the electromagnetic shower can be described as a function
of the {\em radiation length} $X_0$, which is characteristic of the material
\cite{pdg}:
\begin{equation}\label{eq:radiation_length}
X_0 \simeq \frac{716 {\rm g\ cm^{-2}}A}{Z(Z+1)\log{287/\sqrt{Z}}}
\end{equation}
where $A$ is the atomic weight and $Z$ the atomic number. The radiation length
represents the depth in the material at which the electron reduce it energy by a
factor $1/e$:
\begin{equation}\label{eq:radiation_length}
\langle E(x) \rangle = E_0 e^{-\frac{x}{X_0}}
\end{equation}
The radiation length and the critical energy can be used to estimate the depth
at which the maximum number of secondary particles is produced:
\begin{equation}\label{eq:shower_max}
t_{\rm max} \simeq \log{\frac{E_0}{\epsilon}} - 0.5
\end{equation}
where $t_{\rm max}$ is measured in radiation lengths and $E_0$ is the incident
particle energy. The calorimeter thickness containing 95\% of the shower energy
is approximately given by
\begin{equation}\label{eq:95tickness}
t_{95\%} \simeq t_{\rm max}+0.08Z+9.6
\end{equation}
$t_{95\%}$ and $t_{\rm max}$ being expressed in radiation lengths.\\
The electromagnetic shower has a transversal spread due to multiple scattering
away from the shower axis and photon bremsstrahlung. A measurement of the
transverse size of the shower at the critical energy is given by the Moli\`ere radius
\begin{equation}\label{eq:moliere_radius}
R_{\rm M} \simeq 21 {\rm MeV} \frac{X_0}{\epsilon}
\end{equation}
Approximately only the 10\% of the shower energy falls outside the cylinder of
radius $\sim 1 R_{\rm M} $


{ \color{red}
%Tracking insede the caloriemters will be done with Si-trackers covering large
%areas with hundreds of channels


}

